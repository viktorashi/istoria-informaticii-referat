\documentclass[sigplan,screen]{acmart}

%%
%% \BibTeX command to typeset BibTeX logo in the docs
\AtBeginDocument{%
	\providecommand\BibTeX{{%
Bib\TeX}}}

\usepackage{emoji}

% \setemojifont{Apple Color Emoji}

\begin{document}

\title{GNU, Free software and Stallman's dedication}

\author{Stan Ioan-Victor 832}
\email{ioan.victor.stan@stud.ubbcluj.ro}

\begin{teaserfigure}
	\includegraphics[width=200px]{pics/jesus-stallman.jpg}
	\centering
	\caption{RMS in his divine prime}
	\Description{Stallman, in costume as St. IGNUcius, wearing a halo consisting of the platter of an old hard disk drive (Monastir, Tunisia, 2012); description provided by Wikipedia}
	\label{fig:teaser}
\end{teaserfigure}

%% This command processes the author and affiliation and title
%% information and builds the first part of the formatted document.
\maketitle

One of (I feel like) most important key aspects that often goes overlooked in today's "SIGN UP HERE, SUBSCRIBE THERE, ACCEPT MY COOKIESS PlsPLSPLPS" age is privacy and freedom of your personal data or just having technology DO WHAT YOU NEED/WANT IT TO, where now \href{https://www.youtube.com/watch?v=MPyJBJTHyO0}{Facebook requires 1384 hoola hoops}\cite{tantacrul} through which you need to jump just to exercise your right to be forgotten, and where \href{https://www.youtube.com/watch?v=LZzubS1ILTs}{Microsoft shoves ads down your throat}\cite{jakey} when you use their bloated Windows "file search" \textbf{dis-service}\cite{BarraDRM} that forces your prompt through Bing. It's such a shame because these computing things are amazing, they just got handcuffs on them.

But this guy, \textbf{up above} was concerned we might come to this before there even was any code highlighting.

\textbf{Richard Matthew Stallman} (rms) was right in a way, since his cautionary tale all the way back from '97: \href{https://www.gnu.org/philosophy/right-to-read.html}{The Right to Read}\cite{Stallman1997TheRT} has become kind of true over the past years, when services that you technically don't own get re-branded as "products"\cite{product}, but they could rug-pull you at any moment, everything from film\cite{jellyfin}, to of course, as the previous citations pointed out, literature and education.

So how did we get here, and when did my mans realize a change is needed?

\section{That damn symbolics.com}
In the early 80's, Stallman was working on his lil version of the Lisp interpreter, when software conglomerate \href{https://symbolics.com}{symbolics} asked if they could use it for a bit (will give it back unworn promise ongong fr \emoji{crossed-fingers}). You see, Symbolics is a manufacturer of exactly \ldots \textbf{ Lisp machines}, which are exactly what they sound like.

What's important about them is: they are, believe it or not, actually so OG, they were the veryy first \textbf{domain name} ever registered on the internet in 1985. Wayy before ICANN was even a thing and they had to shake hands with "the National Science Foundation (NSF)". ICANN only came out in '95. \cite{national-science-foundation}

Now since Stallman accepted to give them the interpreter via a \textbf{public domain} lisence, they had no qualms about modifying it and keeping it in-house. He then tried to ask for those mods back but he never got them.

His grudge was so intense after that incident that he decided to start the

\section{Free software movement}
In '83 rms launched an OS that would be like Unix, in the sense of its impact and user experience, but not in the sense of how locked-down it was to modifications and its proprietary nature, Unix being owned by AT\&T primarily. He came up with the genious (\verb|\s|) recursive acrnonym of \textbf{G}NU's not Unix. Cuz it wasn't\ldots Unix. "GNU" was so damn good they even named an asteriod after it. \cite{gnu-asteriod}

He basically set out to "free" Unix.

The philosophy behind "free" is the main point of contention for a lot of people learning about the free software movement.

The term broadly reffers to:

\begin{quote}
	"
	software that comes with permission for anyone to use, copy, and/or distribute, either verbatim or with modifications, either gratis or for a fee.
	"
\end{quote}

in their words, putting it best:
\begin{quote}
	"
	you should think of "free" as in "free speech," not as in "free beer."
	"
\end{quote}

Meaning there are 4 \textbf{essential} freedoms the users of such programs benefit from:

\begin{quotation}
	\begin{itemize}
		\item 	The freedom to run the program as you wish, for any purpose. (freedom 0)
		\item 	The freedom to study how the program works, and change it so it does your computing as you wish. Access to the source code is a precondition for this. (freedom 1)
		\item  	The freedom to redistribute copies so you can help others. (freedom 2)
		\item  	The freedom to distribute copies of your modified versions to others. By doing this you can give the whole community a chance to benefit from your changes. Access to the source code is a precondition for this. (freedom 3)
	\end{itemize}
\end{quotation}

Now, armed with the belief that this is the only ethical form of making / distributing software, rms founded the \textbf{Free Software Foundation} (FSF) in 1985, accepting donations and support of even technical kind, in building one of the most important software library in history. Not neccesarily because they initially set out to build them, but because that's what was needed to have a complete, free system. \cite{gnu-and-linux}

You might've heard about a few projects from that library, both end-user focused:
\begin{itemize}
	\item GIMP (\href{https://www.youtube.com/watch?v=pVI_smLgTY0}{pewd's ally now}) \cite{pewds-linux}
	\item LibreOffice (for the broke)
	\item GNOME (for beautiful UI's)\cite{gnome}
	\item emacs
\end{itemize}
And developer-useful
\begin{itemize}
	\item The GNU Compiler Collection (GCC)
	\item GNU Binutils (things like the linker, assembler etc.)
	\item flex
	\item bison
	\item gdb
	\item gscript
	\item BASHHH!! bro, bash
\end{itemize}
And all of these user-space tools were fantastic and sysadmins everywhere were replacing their proprietary counterparts, but the only problem was that they had to be run on top of a non-free kernel and operating system, and they had to be boostrapped by them, usually Unix. They didn't quitee make up a "complete" system just yet.

So they started working on HURD, GNU's free kernel, so they could once and for all distance themselves from non-free software forever. \cite{hurd}

But before they could get it working, one faithful January day in 1992, \textbf{Linus Benedict Torvalds} (yes, like the eggs) put out the release notes for Linux v0.12, his \href{https://www.gnu.org/philosophy/words-to-avoid.en.html#Freemium}{gratis} Kernel, on which he's been working since 1991 (might'a heard of it, no biggie)\cite{linux-init-release-note}, stating:
\begin{quote}
	… I propose that the
	copyright be changed so that it confirms to GNU …
\end{quote}\cite{linux-becomes-free}

\ldots

MONUMENTAL!!!, this is the moment this iconic, out-of-a-superhero-cereal-box figure was created:
\begin{figure}[H]
	\centering
	\includegraphics[width=\columnwidth]{pics/gnu-and-penguin.jpg}
	\Description{Picture of GNU's mascot (a wildebeest) flying alongide Linux's mascot (a pengiun)}
	\caption{a GNU and a lil pengu}
	\label{fig:gnu-linux}
\end{figure}
And so, GNU/Linux "or as I've recently taken to calling it, GNU plus Linux." \cite{gnu-and-linux} was created.

Now what all of these programs had in common was a \textbf{free} licence, not necessarily the in-house produced GNU General Public License, but others that abbided by the aforementioned 4 freedoms. Some include Apache, MIT, BSD, zlib.

\href{https://www.gnu.org/philosophy/categories.html}{Cool breakdown here}. \cite{free-categories}

This doesn't mean there can't be ANY restrictions when distributing modified versions.

In fact, even the copy of the original, free \verb|sample-sigplan.tex| file distributed by ACM that I used for this double-column format had the requirement of

\begin{quote}
	"
	Any modified versions of this file must be renamed
 with new filenames distinct from \verb|sample-sigplan.tex|
	"
\end{quote}
and
\begin{quote}
	"
	This generated file may be distributed as long as the
 original source files, as listed above, are part of the
 same distribution.
	"
\end{quote}
What does the FSF have to say about this?
\begin{quote}
	"
	This sort of requirement is acceptable only if there's a suitable aliasing facility that allows you to specify the original program's name as an alias for the modified version.
	"
\end{quote}
And I mean … I only ever had to change the name of \textbf{this} file and everything else just snapped into place, so checks out.

% ca na, sa ma lase sa compilez
\nocite{*}

\bibliographystyle{ACM-Reference-Format}
\bibliography{biblio}

%% If your work has an appendix, this is the place to put it.
% \appendix

\end{document}
\endinput
